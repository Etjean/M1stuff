
% Default to the notebook output style

    


% Inherit from the specified cell style.




    
\documentclass[11pt]{article}

    
    
    \usepackage[T1]{fontenc}
    % Nicer default font (+ math font) than Computer Modern for most use cases
    \usepackage{mathpazo}

    % Basic figure setup, for now with no caption control since it's done
    % automatically by Pandoc (which extracts ![](path) syntax from Markdown).
    \usepackage{graphicx}
    % We will generate all images so they have a width \maxwidth. This means
    % that they will get their normal width if they fit onto the page, but
    % are scaled down if they would overflow the margins.
    \makeatletter
    \def\maxwidth{\ifdim\Gin@nat@width>\linewidth\linewidth
    \else\Gin@nat@width\fi}
    \makeatother
    \let\Oldincludegraphics\includegraphics
    % Set max figure width to be 80% of text width, for now hardcoded.
    \renewcommand{\includegraphics}[1]{\Oldincludegraphics[width=.8\maxwidth]{#1}}
    % Ensure that by default, figures have no caption (until we provide a
    % proper Figure object with a Caption API and a way to capture that
    % in the conversion process - todo).
    \usepackage{caption}
    \DeclareCaptionLabelFormat{nolabel}{}
    \captionsetup{labelformat=nolabel}

    \usepackage{adjustbox} % Used to constrain images to a maximum size 
    \usepackage{xcolor} % Allow colors to be defined
    \usepackage{enumerate} % Needed for markdown enumerations to work
    \usepackage{geometry} % Used to adjust the document margins
    \usepackage{amsmath} % Equations
    \usepackage{amssymb} % Equations
    \usepackage{textcomp} % defines textquotesingle
    % Hack from http://tex.stackexchange.com/a/47451/13684:
    \AtBeginDocument{%
        \def\PYZsq{\textquotesingle}% Upright quotes in Pygmentized code
    }
    \usepackage{upquote} % Upright quotes for verbatim code
    \usepackage{eurosym} % defines \euro
    \usepackage[mathletters]{ucs} % Extended unicode (utf-8) support
    \usepackage[utf8x]{inputenc} % Allow utf-8 characters in the tex document
    \usepackage{fancyvrb} % verbatim replacement that allows latex
    \usepackage{grffile} % extends the file name processing of package graphics 
                         % to support a larger range 
    % The hyperref package gives us a pdf with properly built
    % internal navigation ('pdf bookmarks' for the table of contents,
    % internal cross-reference links, web links for URLs, etc.)
    \usepackage{hyperref}
    \usepackage{longtable} % longtable support required by pandoc >1.10
    \usepackage{booktabs}  % table support for pandoc > 1.12.2
    \usepackage[inline]{enumitem} % IRkernel/repr support (it uses the enumerate* environment)
    \usepackage[normalem]{ulem} % ulem is needed to support strikethroughs (\sout)
                                % normalem makes italics be italics, not underlines
    

    
    
    % Colors for the hyperref package
    \definecolor{urlcolor}{rgb}{0,.145,.698}
    \definecolor{linkcolor}{rgb}{.71,0.21,0.01}
    \definecolor{citecolor}{rgb}{.12,.54,.11}

    % ANSI colors
    \definecolor{ansi-black}{HTML}{3E424D}
    \definecolor{ansi-black-intense}{HTML}{282C36}
    \definecolor{ansi-red}{HTML}{E75C58}
    \definecolor{ansi-red-intense}{HTML}{B22B31}
    \definecolor{ansi-green}{HTML}{00A250}
    \definecolor{ansi-green-intense}{HTML}{007427}
    \definecolor{ansi-yellow}{HTML}{DDB62B}
    \definecolor{ansi-yellow-intense}{HTML}{B27D12}
    \definecolor{ansi-blue}{HTML}{208FFB}
    \definecolor{ansi-blue-intense}{HTML}{0065CA}
    \definecolor{ansi-magenta}{HTML}{D160C4}
    \definecolor{ansi-magenta-intense}{HTML}{A03196}
    \definecolor{ansi-cyan}{HTML}{60C6C8}
    \definecolor{ansi-cyan-intense}{HTML}{258F8F}
    \definecolor{ansi-white}{HTML}{C5C1B4}
    \definecolor{ansi-white-intense}{HTML}{A1A6B2}

    % commands and environments needed by pandoc snippets
    % extracted from the output of `pandoc -s`
    \providecommand{\tightlist}{%
      \setlength{\itemsep}{0pt}\setlength{\parskip}{0pt}}
    \DefineVerbatimEnvironment{Highlighting}{Verbatim}{commandchars=\\\{\}}
    % Add ',fontsize=\small' for more characters per line
    \newenvironment{Shaded}{}{}
    \newcommand{\KeywordTok}[1]{\textcolor[rgb]{0.00,0.44,0.13}{\textbf{{#1}}}}
    \newcommand{\DataTypeTok}[1]{\textcolor[rgb]{0.56,0.13,0.00}{{#1}}}
    \newcommand{\DecValTok}[1]{\textcolor[rgb]{0.25,0.63,0.44}{{#1}}}
    \newcommand{\BaseNTok}[1]{\textcolor[rgb]{0.25,0.63,0.44}{{#1}}}
    \newcommand{\FloatTok}[1]{\textcolor[rgb]{0.25,0.63,0.44}{{#1}}}
    \newcommand{\CharTok}[1]{\textcolor[rgb]{0.25,0.44,0.63}{{#1}}}
    \newcommand{\StringTok}[1]{\textcolor[rgb]{0.25,0.44,0.63}{{#1}}}
    \newcommand{\CommentTok}[1]{\textcolor[rgb]{0.38,0.63,0.69}{\textit{{#1}}}}
    \newcommand{\OtherTok}[1]{\textcolor[rgb]{0.00,0.44,0.13}{{#1}}}
    \newcommand{\AlertTok}[1]{\textcolor[rgb]{1.00,0.00,0.00}{\textbf{{#1}}}}
    \newcommand{\FunctionTok}[1]{\textcolor[rgb]{0.02,0.16,0.49}{{#1}}}
    \newcommand{\RegionMarkerTok}[1]{{#1}}
    \newcommand{\ErrorTok}[1]{\textcolor[rgb]{1.00,0.00,0.00}{\textbf{{#1}}}}
    \newcommand{\NormalTok}[1]{{#1}}
    
    % Additional commands for more recent versions of Pandoc
    \newcommand{\ConstantTok}[1]{\textcolor[rgb]{0.53,0.00,0.00}{{#1}}}
    \newcommand{\SpecialCharTok}[1]{\textcolor[rgb]{0.25,0.44,0.63}{{#1}}}
    \newcommand{\VerbatimStringTok}[1]{\textcolor[rgb]{0.25,0.44,0.63}{{#1}}}
    \newcommand{\SpecialStringTok}[1]{\textcolor[rgb]{0.73,0.40,0.53}{{#1}}}
    \newcommand{\ImportTok}[1]{{#1}}
    \newcommand{\DocumentationTok}[1]{\textcolor[rgb]{0.73,0.13,0.13}{\textit{{#1}}}}
    \newcommand{\AnnotationTok}[1]{\textcolor[rgb]{0.38,0.63,0.69}{\textbf{\textit{{#1}}}}}
    \newcommand{\CommentVarTok}[1]{\textcolor[rgb]{0.38,0.63,0.69}{\textbf{\textit{{#1}}}}}
    \newcommand{\VariableTok}[1]{\textcolor[rgb]{0.10,0.09,0.49}{{#1}}}
    \newcommand{\ControlFlowTok}[1]{\textcolor[rgb]{0.00,0.44,0.13}{\textbf{{#1}}}}
    \newcommand{\OperatorTok}[1]{\textcolor[rgb]{0.40,0.40,0.40}{{#1}}}
    \newcommand{\BuiltInTok}[1]{{#1}}
    \newcommand{\ExtensionTok}[1]{{#1}}
    \newcommand{\PreprocessorTok}[1]{\textcolor[rgb]{0.74,0.48,0.00}{{#1}}}
    \newcommand{\AttributeTok}[1]{\textcolor[rgb]{0.49,0.56,0.16}{{#1}}}
    \newcommand{\InformationTok}[1]{\textcolor[rgb]{0.38,0.63,0.69}{\textbf{\textit{{#1}}}}}
    \newcommand{\WarningTok}[1]{\textcolor[rgb]{0.38,0.63,0.69}{\textbf{\textit{{#1}}}}}
    
    
    % Define a nice break command that doesn't care if a line doesn't already
    % exist.
    \def\br{\hspace*{\fill} \\* }
    % Math Jax compatability definitions
    \def\gt{>}
    \def\lt{<}
    % Document parameters
    \title{Projet Protein Docking  - 1GPW}
    
    
    

    % Pygments definitions
    
\makeatletter
\def\PY@reset{\let\PY@it=\relax \let\PY@bf=\relax%
    \let\PY@ul=\relax \let\PY@tc=\relax%
    \let\PY@bc=\relax \let\PY@ff=\relax}
\def\PY@tok#1{\csname PY@tok@#1\endcsname}
\def\PY@toks#1+{\ifx\relax#1\empty\else%
    \PY@tok{#1}\expandafter\PY@toks\fi}
\def\PY@do#1{\PY@bc{\PY@tc{\PY@ul{%
    \PY@it{\PY@bf{\PY@ff{#1}}}}}}}
\def\PY#1#2{\PY@reset\PY@toks#1+\relax+\PY@do{#2}}

\expandafter\def\csname PY@tok@w\endcsname{\def\PY@tc##1{\textcolor[rgb]{0.73,0.73,0.73}{##1}}}
\expandafter\def\csname PY@tok@c\endcsname{\let\PY@it=\textit\def\PY@tc##1{\textcolor[rgb]{0.25,0.50,0.50}{##1}}}
\expandafter\def\csname PY@tok@cp\endcsname{\def\PY@tc##1{\textcolor[rgb]{0.74,0.48,0.00}{##1}}}
\expandafter\def\csname PY@tok@k\endcsname{\let\PY@bf=\textbf\def\PY@tc##1{\textcolor[rgb]{0.00,0.50,0.00}{##1}}}
\expandafter\def\csname PY@tok@kp\endcsname{\def\PY@tc##1{\textcolor[rgb]{0.00,0.50,0.00}{##1}}}
\expandafter\def\csname PY@tok@kt\endcsname{\def\PY@tc##1{\textcolor[rgb]{0.69,0.00,0.25}{##1}}}
\expandafter\def\csname PY@tok@o\endcsname{\def\PY@tc##1{\textcolor[rgb]{0.40,0.40,0.40}{##1}}}
\expandafter\def\csname PY@tok@ow\endcsname{\let\PY@bf=\textbf\def\PY@tc##1{\textcolor[rgb]{0.67,0.13,1.00}{##1}}}
\expandafter\def\csname PY@tok@nb\endcsname{\def\PY@tc##1{\textcolor[rgb]{0.00,0.50,0.00}{##1}}}
\expandafter\def\csname PY@tok@nf\endcsname{\def\PY@tc##1{\textcolor[rgb]{0.00,0.00,1.00}{##1}}}
\expandafter\def\csname PY@tok@nc\endcsname{\let\PY@bf=\textbf\def\PY@tc##1{\textcolor[rgb]{0.00,0.00,1.00}{##1}}}
\expandafter\def\csname PY@tok@nn\endcsname{\let\PY@bf=\textbf\def\PY@tc##1{\textcolor[rgb]{0.00,0.00,1.00}{##1}}}
\expandafter\def\csname PY@tok@ne\endcsname{\let\PY@bf=\textbf\def\PY@tc##1{\textcolor[rgb]{0.82,0.25,0.23}{##1}}}
\expandafter\def\csname PY@tok@nv\endcsname{\def\PY@tc##1{\textcolor[rgb]{0.10,0.09,0.49}{##1}}}
\expandafter\def\csname PY@tok@no\endcsname{\def\PY@tc##1{\textcolor[rgb]{0.53,0.00,0.00}{##1}}}
\expandafter\def\csname PY@tok@nl\endcsname{\def\PY@tc##1{\textcolor[rgb]{0.63,0.63,0.00}{##1}}}
\expandafter\def\csname PY@tok@ni\endcsname{\let\PY@bf=\textbf\def\PY@tc##1{\textcolor[rgb]{0.60,0.60,0.60}{##1}}}
\expandafter\def\csname PY@tok@na\endcsname{\def\PY@tc##1{\textcolor[rgb]{0.49,0.56,0.16}{##1}}}
\expandafter\def\csname PY@tok@nt\endcsname{\let\PY@bf=\textbf\def\PY@tc##1{\textcolor[rgb]{0.00,0.50,0.00}{##1}}}
\expandafter\def\csname PY@tok@nd\endcsname{\def\PY@tc##1{\textcolor[rgb]{0.67,0.13,1.00}{##1}}}
\expandafter\def\csname PY@tok@s\endcsname{\def\PY@tc##1{\textcolor[rgb]{0.73,0.13,0.13}{##1}}}
\expandafter\def\csname PY@tok@sd\endcsname{\let\PY@it=\textit\def\PY@tc##1{\textcolor[rgb]{0.73,0.13,0.13}{##1}}}
\expandafter\def\csname PY@tok@si\endcsname{\let\PY@bf=\textbf\def\PY@tc##1{\textcolor[rgb]{0.73,0.40,0.53}{##1}}}
\expandafter\def\csname PY@tok@se\endcsname{\let\PY@bf=\textbf\def\PY@tc##1{\textcolor[rgb]{0.73,0.40,0.13}{##1}}}
\expandafter\def\csname PY@tok@sr\endcsname{\def\PY@tc##1{\textcolor[rgb]{0.73,0.40,0.53}{##1}}}
\expandafter\def\csname PY@tok@ss\endcsname{\def\PY@tc##1{\textcolor[rgb]{0.10,0.09,0.49}{##1}}}
\expandafter\def\csname PY@tok@sx\endcsname{\def\PY@tc##1{\textcolor[rgb]{0.00,0.50,0.00}{##1}}}
\expandafter\def\csname PY@tok@m\endcsname{\def\PY@tc##1{\textcolor[rgb]{0.40,0.40,0.40}{##1}}}
\expandafter\def\csname PY@tok@gh\endcsname{\let\PY@bf=\textbf\def\PY@tc##1{\textcolor[rgb]{0.00,0.00,0.50}{##1}}}
\expandafter\def\csname PY@tok@gu\endcsname{\let\PY@bf=\textbf\def\PY@tc##1{\textcolor[rgb]{0.50,0.00,0.50}{##1}}}
\expandafter\def\csname PY@tok@gd\endcsname{\def\PY@tc##1{\textcolor[rgb]{0.63,0.00,0.00}{##1}}}
\expandafter\def\csname PY@tok@gi\endcsname{\def\PY@tc##1{\textcolor[rgb]{0.00,0.63,0.00}{##1}}}
\expandafter\def\csname PY@tok@gr\endcsname{\def\PY@tc##1{\textcolor[rgb]{1.00,0.00,0.00}{##1}}}
\expandafter\def\csname PY@tok@ge\endcsname{\let\PY@it=\textit}
\expandafter\def\csname PY@tok@gs\endcsname{\let\PY@bf=\textbf}
\expandafter\def\csname PY@tok@gp\endcsname{\let\PY@bf=\textbf\def\PY@tc##1{\textcolor[rgb]{0.00,0.00,0.50}{##1}}}
\expandafter\def\csname PY@tok@go\endcsname{\def\PY@tc##1{\textcolor[rgb]{0.53,0.53,0.53}{##1}}}
\expandafter\def\csname PY@tok@gt\endcsname{\def\PY@tc##1{\textcolor[rgb]{0.00,0.27,0.87}{##1}}}
\expandafter\def\csname PY@tok@err\endcsname{\def\PY@bc##1{\setlength{\fboxsep}{0pt}\fcolorbox[rgb]{1.00,0.00,0.00}{1,1,1}{\strut ##1}}}
\expandafter\def\csname PY@tok@kc\endcsname{\let\PY@bf=\textbf\def\PY@tc##1{\textcolor[rgb]{0.00,0.50,0.00}{##1}}}
\expandafter\def\csname PY@tok@kd\endcsname{\let\PY@bf=\textbf\def\PY@tc##1{\textcolor[rgb]{0.00,0.50,0.00}{##1}}}
\expandafter\def\csname PY@tok@kn\endcsname{\let\PY@bf=\textbf\def\PY@tc##1{\textcolor[rgb]{0.00,0.50,0.00}{##1}}}
\expandafter\def\csname PY@tok@kr\endcsname{\let\PY@bf=\textbf\def\PY@tc##1{\textcolor[rgb]{0.00,0.50,0.00}{##1}}}
\expandafter\def\csname PY@tok@bp\endcsname{\def\PY@tc##1{\textcolor[rgb]{0.00,0.50,0.00}{##1}}}
\expandafter\def\csname PY@tok@fm\endcsname{\def\PY@tc##1{\textcolor[rgb]{0.00,0.00,1.00}{##1}}}
\expandafter\def\csname PY@tok@vc\endcsname{\def\PY@tc##1{\textcolor[rgb]{0.10,0.09,0.49}{##1}}}
\expandafter\def\csname PY@tok@vg\endcsname{\def\PY@tc##1{\textcolor[rgb]{0.10,0.09,0.49}{##1}}}
\expandafter\def\csname PY@tok@vi\endcsname{\def\PY@tc##1{\textcolor[rgb]{0.10,0.09,0.49}{##1}}}
\expandafter\def\csname PY@tok@vm\endcsname{\def\PY@tc##1{\textcolor[rgb]{0.10,0.09,0.49}{##1}}}
\expandafter\def\csname PY@tok@sa\endcsname{\def\PY@tc##1{\textcolor[rgb]{0.73,0.13,0.13}{##1}}}
\expandafter\def\csname PY@tok@sb\endcsname{\def\PY@tc##1{\textcolor[rgb]{0.73,0.13,0.13}{##1}}}
\expandafter\def\csname PY@tok@sc\endcsname{\def\PY@tc##1{\textcolor[rgb]{0.73,0.13,0.13}{##1}}}
\expandafter\def\csname PY@tok@dl\endcsname{\def\PY@tc##1{\textcolor[rgb]{0.73,0.13,0.13}{##1}}}
\expandafter\def\csname PY@tok@s2\endcsname{\def\PY@tc##1{\textcolor[rgb]{0.73,0.13,0.13}{##1}}}
\expandafter\def\csname PY@tok@sh\endcsname{\def\PY@tc##1{\textcolor[rgb]{0.73,0.13,0.13}{##1}}}
\expandafter\def\csname PY@tok@s1\endcsname{\def\PY@tc##1{\textcolor[rgb]{0.73,0.13,0.13}{##1}}}
\expandafter\def\csname PY@tok@mb\endcsname{\def\PY@tc##1{\textcolor[rgb]{0.40,0.40,0.40}{##1}}}
\expandafter\def\csname PY@tok@mf\endcsname{\def\PY@tc##1{\textcolor[rgb]{0.40,0.40,0.40}{##1}}}
\expandafter\def\csname PY@tok@mh\endcsname{\def\PY@tc##1{\textcolor[rgb]{0.40,0.40,0.40}{##1}}}
\expandafter\def\csname PY@tok@mi\endcsname{\def\PY@tc##1{\textcolor[rgb]{0.40,0.40,0.40}{##1}}}
\expandafter\def\csname PY@tok@il\endcsname{\def\PY@tc##1{\textcolor[rgb]{0.40,0.40,0.40}{##1}}}
\expandafter\def\csname PY@tok@mo\endcsname{\def\PY@tc##1{\textcolor[rgb]{0.40,0.40,0.40}{##1}}}
\expandafter\def\csname PY@tok@ch\endcsname{\let\PY@it=\textit\def\PY@tc##1{\textcolor[rgb]{0.25,0.50,0.50}{##1}}}
\expandafter\def\csname PY@tok@cm\endcsname{\let\PY@it=\textit\def\PY@tc##1{\textcolor[rgb]{0.25,0.50,0.50}{##1}}}
\expandafter\def\csname PY@tok@cpf\endcsname{\let\PY@it=\textit\def\PY@tc##1{\textcolor[rgb]{0.25,0.50,0.50}{##1}}}
\expandafter\def\csname PY@tok@c1\endcsname{\let\PY@it=\textit\def\PY@tc##1{\textcolor[rgb]{0.25,0.50,0.50}{##1}}}
\expandafter\def\csname PY@tok@cs\endcsname{\let\PY@it=\textit\def\PY@tc##1{\textcolor[rgb]{0.25,0.50,0.50}{##1}}}

\def\PYZbs{\char`\\}
\def\PYZus{\char`\_}
\def\PYZob{\char`\{}
\def\PYZcb{\char`\}}
\def\PYZca{\char`\^}
\def\PYZam{\char`\&}
\def\PYZlt{\char`\<}
\def\PYZgt{\char`\>}
\def\PYZsh{\char`\#}
\def\PYZpc{\char`\%}
\def\PYZdl{\char`\$}
\def\PYZhy{\char`\-}
\def\PYZsq{\char`\'}
\def\PYZdq{\char`\"}
\def\PYZti{\char`\~}
% for compatibility with earlier versions
\def\PYZat{@}
\def\PYZlb{[}
\def\PYZrb{]}
\makeatother


    % Exact colors from NB
    \definecolor{incolor}{rgb}{0.0, 0.0, 0.5}
    \definecolor{outcolor}{rgb}{0.545, 0.0, 0.0}



    
    % Prevent overflowing lines due to hard-to-break entities
    \sloppy 
    % Setup hyperref package
    \hypersetup{
      breaklinks=true,  % so long urls are correctly broken across lines
      colorlinks=true,
      urlcolor=urlcolor,
      linkcolor=linkcolor,
      citecolor=citecolor,
      }
    % Slightly bigger margins than the latex defaults
    
    \geometry{verbose,tmargin=1in,bmargin=1in,lmargin=1in,rmargin=1in}
    
    

    \begin{document}
    
    
    \maketitle
    
    

    
    Etienne JEAN

    \section{Introduction}\label{introduction}

The \emph{Imidazole Glycerol Phosphate Synthase} (IGPS) is a
heterodimeric protein of the organism \emph{Thermotoga maritima}
composed of two subunits : a lyase enzyme named \emph{hisF} and a
transferase enzyme named \emph{hisH}. IGPS is involved in the histidine
biosynthesis pathway by catalyzing the conversion of PRFAR and glutamine
to IGP, AICAR and glutamate. The HisF subunit catalyzes the cyclization
activity that produces IGP and AICAR from PRFAR using the ammonia
provided by the HisH subunit. It is a protein found in the cytoplasm of
the cells.

The goal of this study is to simulate a docking \emph{in silico} between
the two unbound subunits of IGPS, using both a rigid-body technique and
an energy-based technique, and finally evaluate and discuss the quality
of the docking results.

    \section{Material \& Methods}\label{material-methods}

\subsection{Structures}\label{structures}

The structures of the unbound proteins and of the complex were taken
from the \emph{Protein Data Bank} (PDB).\\
- Unbound hisF subunit (253 residues) : code 1THF, one chain D
https://www.rcsb.org/structure/1THF\\
- Unbound hisH subunit (201 residues) : code 1K9V, one chain F
https://www.rcsb.org/structure/1K9V\\
- Complex IGPS : code 1GPW, chains A and B
https://www.rcsb.org/structure/1GPW

\subsection{Z-Dock}\label{z-dock}

From these structures, two docking were computed using two different
methods. The first method used was a rigid-body docking, based on
geometry complementarity of the two subunits and fast fourier transform
(FFT) based search, and it was realised with the program Z-Dock 2.1.

The docking was conducted as follows :\\
1. \textbf{Preparation of the input files by only keeping the "ATOM"
lines of the pdb files :}

\begin{verbatim}
cat 1THF.pdb | grep ATOM > 1THFatom.pdb
cat 1K9V.pdb | grep ATOM > 1K9Vatom.pdb
\end{verbatim}

As there is only one chain in each file, there were no need to edit them
anymore.\\

\begin{enumerate}
\def\labelenumi{\arabic{enumi}.}
\setcounter{enumi}{1}
\item
  \textbf{Preparation of Ligand and Receptor files.}\\
  From this point, hisH will be considered the receptor and hisF will be
  considered the ligand. Note that this choice is a little awkward, as
  hisF is a slightly bigger protein than hisH (253 residues and 201),
  but its only consequence was to increase the calculation time.\\
  Thus, the receptor and ligand files were created using the command
  \texttt{mark\_sur}.

\begin{verbatim}
mark_sur 1K9Vatom.pdb 1GPW_r.pdb
mark_sur 1THFatom.pdb 1GPW_l.pdb
\end{verbatim}
\item
  \textbf{Docking and PDB generation.} Finally, the docking was computed
  using \texttt{zdock}.

\begin{verbatim}
zdock -R 1GPW_r.pdb -L 1GPW_l.pdb -o 1GPW_zdock.out
\end{verbatim}

  The output file contains 2000 docking poses, sorted from top to bottom
  according to their Z-Dock score. Only the first ten were kept, thus
  supposedly the best docking, in a modified output file, and the PDB
  outputs were then generated.

\begin{verbatim}
head -n 14 1GPW_zdock.out > 1GPW_zdock_10.out
create.pl 1_zdock_10.out
\end{verbatim}
\end{enumerate}

\subsection{PyDock}\label{pydock}

The second method used to dock hisH and hisF was an energy-based method,
using PyDock 3.0.

\begin{enumerate}
\def\labelenumi{\arabic{enumi}.}
\item
  \textbf{Initial setup}\\
  In order to generate the pdb files that pydock uses for docking, a
  \texttt{1gpw\_pydock.ini} file was made with the following text :

\begin{verbatim}
[receptor]  
pdb = 1K9Vatom_scwrl.pdb  
mol = F  
newmol = F    
[ligand]  
pdb = 1THFatom.pdb  
mol = D  
newmol = D  
\end{verbatim}

  Note that the pdb file 1K9V missed some atoms, so the program
  \texttt{scwrl3} was used to fix this issue beforehand.

\begin{verbatim}
scwrl3 -i 1K9Vatom.pdb -o 1K9Vatom_scwrl.pdb
\end{verbatim}

  Finally, the agument \texttt{setup} was passed to pydock to create the
  receptor and ligand files.

\begin{verbatim}
pyDock3 1gpw_pydock setup
\end{verbatim}
\item
  \textbf{Rigid-body docking as a base for PyDock}\\
  Rigid-body docking orientations were generated using the zdock
  algorithm as base positions for later energy calculations.

\begin{verbatim}
pyDock3 1gpw_pydock zdock
\end{verbatim}

  The output of that command creates a file named
  \texttt{1gpw\_pydock.zdock}.\\
\item
  \textbf{Computation of translation and rotation matrix}\\
  The translation and rotation matrix need to be computed from the zdock
  output file.

\begin{verbatim}
pyDock3 1gpw_pydock rotzdock
\end{verbatim}

  The output of that command creates a file named
  \texttt{1gpw\_pydock.rot}. For calculations time reasons, only 100
  conformations were kept from this file (out of 2000) for the next
  step.\\
\item
  \textbf{Computation of energy scores}\\
  Finally, the energy were computed to score and rank all docking poses,
  using the argument \texttt{dockser}.

\begin{verbatim}
pyDock3 1gpw_pydock dockser > dockser.log
\end{verbatim}

  The output of that command creates the final file named
  \texttt{1gpw\_pydock.ene}.\\
\end{enumerate}

\subsection{PyMOL}\label{pymol}

The visualisation and analysis of the results was done with PyMOL. After
loading both the reference structure of the complexed hisF and hisH (pdb
1GPW) and the results of the docking, the structure of the receptors
were aligned together, and the Root Mean Square Deviation of the ligands
(L-RMSD) were computed to evaluate the distance of the docking proposed
by Zdock and FTdock to the reference docking pose.

    \section{Results \& Discussion}\label{results-discussion}

\subsection{Z-Dock}\label{z-dock}

The results of the z-dock docking and their ligand-RMSD are listed in
the table below.

\begin{longtable}[]{@{}ccc@{}}
\toprule
\begin{minipage}[b]{0.19\columnwidth}\centering\strut
Rank in Zdock scoring\strut
\end{minipage} & \begin{minipage}[b]{0.22\columnwidth}\centering\strut
Zdock score\strut
\end{minipage} & \begin{minipage}[b]{0.49\columnwidth}\centering\strut
Ligand-RMSD compared to the reference structure (in Angströms)\strut
\end{minipage}\tabularnewline
\midrule
\endhead
\begin{minipage}[t]{0.19\columnwidth}\centering\strut
\textbf{1}\strut
\end{minipage} & \begin{minipage}[t]{0.22\columnwidth}\centering\strut
\textbf{18.34}\strut
\end{minipage} & \begin{minipage}[t]{0.49\columnwidth}\centering\strut
\textbf{14.886}\strut
\end{minipage}\tabularnewline
\begin{minipage}[t]{0.19\columnwidth}\centering\strut
\textbf{2}\strut
\end{minipage} & \begin{minipage}[t]{0.22\columnwidth}\centering\strut
\textbf{18.32}\strut
\end{minipage} & \begin{minipage}[t]{0.49\columnwidth}\centering\strut
\textbf{14.660}\strut
\end{minipage}\tabularnewline
\begin{minipage}[t]{0.19\columnwidth}\centering\strut
3\strut
\end{minipage} & \begin{minipage}[t]{0.22\columnwidth}\centering\strut
17.60\strut
\end{minipage} & \begin{minipage}[t]{0.49\columnwidth}\centering\strut
21.302\strut
\end{minipage}\tabularnewline
\begin{minipage}[t]{0.19\columnwidth}\centering\strut
4\strut
\end{minipage} & \begin{minipage}[t]{0.22\columnwidth}\centering\strut
17.48\strut
\end{minipage} & \begin{minipage}[t]{0.49\columnwidth}\centering\strut
15.545\strut
\end{minipage}\tabularnewline
\begin{minipage}[t]{0.19\columnwidth}\centering\strut
5\strut
\end{minipage} & \begin{minipage}[t]{0.22\columnwidth}\centering\strut
17.36\strut
\end{minipage} & \begin{minipage}[t]{0.49\columnwidth}\centering\strut
15.963\strut
\end{minipage}\tabularnewline
\begin{minipage}[t]{0.19\columnwidth}\centering\strut
6\strut
\end{minipage} & \begin{minipage}[t]{0.22\columnwidth}\centering\strut
16.88\strut
\end{minipage} & \begin{minipage}[t]{0.49\columnwidth}\centering\strut
19.840\strut
\end{minipage}\tabularnewline
\begin{minipage}[t]{0.19\columnwidth}\centering\strut
7\strut
\end{minipage} & \begin{minipage}[t]{0.22\columnwidth}\centering\strut
16.64\strut
\end{minipage} & \begin{minipage}[t]{0.49\columnwidth}\centering\strut
20.133\strut
\end{minipage}\tabularnewline
\begin{minipage}[t]{0.19\columnwidth}\centering\strut
8\strut
\end{minipage} & \begin{minipage}[t]{0.22\columnwidth}\centering\strut
16.56\strut
\end{minipage} & \begin{minipage}[t]{0.49\columnwidth}\centering\strut
15.243\strut
\end{minipage}\tabularnewline
\begin{minipage}[t]{0.19\columnwidth}\centering\strut
\textbf{9}\strut
\end{minipage} & \begin{minipage}[t]{0.22\columnwidth}\centering\strut
\textbf{16.54}\strut
\end{minipage} & \begin{minipage}[t]{0.49\columnwidth}\centering\strut
\textbf{13.980}\strut
\end{minipage}\tabularnewline
\begin{minipage}[t]{0.19\columnwidth}\centering\strut
10\strut
\end{minipage} & \begin{minipage}[t]{0.22\columnwidth}\centering\strut
16.38\strut
\end{minipage} & \begin{minipage}[t]{0.49\columnwidth}\centering\strut
15.989\strut
\end{minipage}\tabularnewline
\bottomrule
\end{longtable}

All L-RMSD range from 14 Angströms to 21 Angströms. The docking show 3
remarkable docking positions from the top 10 scores : Docking 1 and 2
have both a high z-dock score and a low L-RMSD. Docking 9 has a lower
score but has the best L-RMSD compared to the reference complex, which
still make it an interesting (maybe the most) docking pose.

\subsection{PyDock}\label{pydock}

The PyDock energy calculations and rankings are listed in the table
below.

\begin{longtable}[]{@{}cccccc@{}}
\toprule
Zdock rank & E.electrostatic & E.desolvatation & E.VDW & E.Total &
RANK\tabularnewline
\midrule
\endhead
97 & -18.843 & -3.831 & 44.166 & -18.257 & 1\tabularnewline
83 & -33.248 & 15.107 & 70.255 & -11.116 & 2\tabularnewline
63 & -20.052 & 3.819 & 70.430 & -9.190 & 3\tabularnewline
14 & -18.476 & 7.023 & 56.848 & -5.768 & 4\tabularnewline
9 & -23.479 & 15.693 & 26.901 & -5.096 & 5\tabularnewline
90 & -39.603 & 31.659 & 68.477 & -1.096 & 6\tabularnewline
88 & -24.285 & 20.566 & 41.517 & 0.432 & 7\tabularnewline
3 & -45.283 & 22.002 & 240.869 & 0.806 & 8\tabularnewline
22 & -31.099 & 22.158 & 129.200 & 3.979 & 9\tabularnewline
77 & -38.364 & 14.441 & 282.226 & 4.299 & 10\tabularnewline
\bottomrule
\end{longtable}

Once again, only the 10 best results are shown. The total energies of
the docking range from -18 kcal/mol to +4 kcal/mol. What is the most
surprising is the complete absence of correlation between the Z-dock
ranking and the Pydock ranking, since the best docking according to
pydock is almost the lowest ranked of the 100 zdock subset that was
taken (ranked 97/100 by pydock). This emphasizes the limitations of
rigid-body docking methods. Those methods are rapid and simple to
execute, but the resulting docking often lack precision compared to
other methods.\\
The docking ranked 9th by Zdock is here ranked 5th by pydock. Thus this
docking might be considered one of the best among all the results,
because both his L-RMSD and its total energy are low.

    \section{Conclusion}\label{conclusion}

In this study, thanks to two different and complementary docking
methods, a rigid-body docking and an energy-based ranking, a hundred
docking poses of the lyase subunit hisF and the transferase subunit hisH
from the Imidazole glycerol phosphate synthase were generated. One
position in particliar was found in the top 10 of both methods, and
showed both a low ligand-RMSD and a low docking energy.\\
This study also allowed to compare both techniques and there quality in
protein docking : the rigid-body docking allows to generate thousands of
positionings quickly and simply, but its score function lacks a lot of
precision and is often unrelated with what really happens \emph{in
vivo}. The energy-based method however allows to improve a lot the
scoring function. Thus those two methods combined allow to return a set
of dockings of pretty good precision.


    % Add a bibliography block to the postdoc
    
    
    
    \end{document}
